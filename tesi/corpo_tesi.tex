% 12pt: grandezza carattere
% a4paper: formato a4
% openright: apre i capitoli a destra
% twoside: serve per fare un documento fronteretro
% report: stile tesi (oppure book)
\documentclass[12pt,a4paper,openright,twoside]{report}

\usepackage[italian]{babel} 	% libreria per scrivere in italiano
\usepackage[latin1]{inputenc}   % libreria per accettare i caratteri digitati da tastiera come 'è' o 'à'
\usepackage{fancyhdr}   		% libreria per impostare il documento
\usepackage{indentfirst}    	% libreria per avere l'indentazione all'inizio dei capitoli
%\usepackage{showkeys}  		% libreria per mostrare le etichette
\usepackage{graphicx}   		% libreria per inserire grafici
\usepackage{newlfont}   		% libreria per utilizzare font particolari ad esempio \textsc{}
% librerie matematiche
\usepackage{amssymb}
\usepackage{amsmath}
\usepackage{latexsym}
\usepackage{amsthm}

\oddsidemargin=30pt \evensidemargin=20pt	% impostano i margini
%\hyphenation{sil-la-ba-zio-ne pa-ren-te-si}	% serve per la sillabazione: tra parentesi vanno inserite come nell'esempio le parole che latex non riesce a tagliare nel modo giusto andando a capo

% comandi per l'impostazione della pagina, vedi il manuale della libreria fancyhdr per ulteriori delucidazioni
\pagestyle{fancy}\addtolength{\headwidth}{20pt}
\renewcommand{\chaptermark}[1]{\markboth{\thechapter.\ #1}{}}
\renewcommand{\sectionmark}[1]{\markright{\thesection \ #1}{}}
\rhead[\fancyplain{}{\bfseries\leftmark}]{\fancyplain{}{\bfseries\thepage}}
\cfoot{}

\linespread{1.3}	% comando per impostare l'interlinea



\begin{document}


\pagenumbering{roman}	% serve per mettere i numeri romani

\chapter*{Introduzione}		% crea l'introduzione (un capitolo non numerato)

% imposta l'intestazione di pagina
\rhead[\fancyplain{}{\bfseries
INTRODUZIONE}]{\fancyplain{}{\bfseries\thepage}}
\lhead[\fancyplain{}{\bfseries\thepage}]{\fancyplain{}{\bfseries
INTRODUZIONE}}

\addcontentsline{toc}{chapter}{Introduzione}	% aggiunge la voce Introduzione nell'indice

Questa \`e l'introduzione.

\clearpage{\pagestyle{empty}\cleardoublepage}	% non numera l'ultima pagina sinistra


\tableofcontents	% crea l'indice

% imposta l'intestazione di pagina
\rhead[\fancyplain{}{\bfseries\leftmark}]{\fancyplain{}{\bfseries\thepage}}
\lhead[\fancyplain{}{\bfseries\thepage}]{\fancyplain{}{\bfseries
INDICE}}

\clearpage{\pagestyle{empty}\cleardoublepage}	% non numera l'ultima pagina sinistra


%\listoffigures		% crea l'elenco delle figure

%\clearpage{\pagestyle{empty}\cleardoublepage}		% non numera l'ultima pagina sinistra


%\listoftables		% crea l'elenco delle tabelle

%\clearpage{\pagestyle{empty}\cleardoublepage}		% non numera l'ultima pagina sinistra


\chapter{Primo Capitolo}		% crea il capitolo

\lhead[\fancyplain{}{\bfseries\thepage}]{\fancyplain{}{\bfseries\rightmark}}	% imposta l'intestazione di pagina

\pagenumbering{arabic}		% mette i numeri arabi

Questo \`e il primo capitolo.

\clearpage{\pagestyle{empty}\cleardoublepage}		% non numera l'ultima pagina sinistra


\chapter*{Conclusioni}		% per fare le conclusioni

% imposta l'intestazione di pagina
\rhead[\fancyplain{}{\bfseries
CONCLUSIONI}]{\fancyplain{}{\bfseries\thepage}}
\lhead[\fancyplain{}{\bfseries\thepage}]{\fancyplain{}{\bfseries
CONCLUSIONI}}

\addcontentsline{toc}{chapter}{Conclusioni}		% aggiunge la voce Conclusioni nell'indice

Queste sono le conclusioni.\\
In queste conclusioni voglio fare un riferimento alla bibliografia: questo \`e il mio riferimento \cite{K3,K4}.


%\begin{thebibliography}{90}		% crea l'ambiente bibliografia

%\rhead[\fancyplain{}{\bfseries \leftmark}]{\fancyplain{}{\bfseries
%\thepage}}

%\addcontentsline{toc}{chapter}{Bibliografia}	% aggiunge la voce Bibliografia nell'indice
%%\addcontentsline{toc}{chapter}{\numberline{}{Bibliografia}}	% provare anche questo comando

%\bibitem{K1} Primo oggetto bibliografia.
%\bibitem{K2} Secondo oggetto bibliografia.
%\bibitem{K3} Terzo oggetto bibliografia.
%\bibitem{K4} Quarto oggetto bibliografia.

%\end{thebibliography}

\clearpage{\pagestyle{empty}\cleardoublepage}	% non numera l'ultima pagina sinistra


\chapter*{Ringraziamenti}

\thispagestyle{empty}

Qui possiamo ringraziare il mondo intero!!!!!!!!!!\\
Ovviamente solo se uno vuole, non \`e obbligatorio.


\end{document}
