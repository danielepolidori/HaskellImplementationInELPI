% 12pt: grandezza carattere
% a4paper: formato a4
% openright: apre i capitoli a destra
% twoside: serve per fare un documento fronteretro
% report: stile tesi (oppure book)
\documentclass[12pt,a4paper,openright,twoside]{report}

\usepackage[italian]{babel} 	% libreria per scrivere in italiano
\usepackage[utf8]{inputenc}   % libreria per accettare i caratteri digitati da tastiera come 'è' o 'à'
\usepackage{fancyhdr}   		% libreria per impostare il documento
\usepackage{indentfirst}    	% libreria per avere l'indentazione all'inizio dei capitoli
%\usepackage{showkeys}  		% libreria per mostrare le etichette
\usepackage{graphicx}   		% libreria per inserire grafici
\usepackage{newlfont}   		% libreria per utilizzare font particolari ad esempio \textsc{}
% librerie matematiche
\usepackage{amssymb}
\usepackage{amsmath}
\usepackage{latexsym}
\usepackage{amsthm}

\oddsidemargin=30pt \evensidemargin=20pt	% impostano i margini
%\hyphenation{sil-la-ba-zio-ne pa-ren-te-si}	% serve per la sillabazione: tra parentesi vanno inserite come nell'esempio le parole che latex non riesce a tagliare nel modo giusto andando a capo

% comandi per l'impostazione della pagina, vedi il manuale della libreria fancyhdr per ulteriori delucidazioni
\pagestyle{fancy}\addtolength{\headwidth}{20pt}
\renewcommand{\chaptermark}[1]{\markboth{\thechapter.\ #1}{}}
\renewcommand{\sectionmark}[1]{\markright{\thesection \ #1}{}}
\rhead[\fancyplain{}{\bfseries\leftmark}]{\fancyplain{}{\bfseries\thepage}}
\cfoot{}

\linespread{1.3}	% comando per impostare l'interlinea



\begin{document}


\pagenumbering{arabic}		% mette i numeri arabi


\tableofcontents	% crea l'indice

% imposta l'intestazione di pagina
\rhead[\fancyplain{}{\bfseries\leftmark}]{\fancyplain{}{\bfseries\thepage}}
\lhead[\fancyplain{}{\bfseries\thepage}]{\fancyplain{}{\bfseries
INDICE}}

\clearpage{\pagestyle{empty}\cleardoublepage}	% non numera l'ultima pagina sinistra


%\listoffigures		% crea l'elenco delle figure

%\clearpage{\pagestyle{empty}\cleardoublepage}		% non numera l'ultima pagina sinistra


%\listoftables		% crea l'elenco delle tabelle

%\clearpage{\pagestyle{empty}\cleardoublepage}		% non numera l'ultima pagina sinistra


\chapter{Introduzione}		% crea l'introduzione

% imposta l'intestazione di pagina
\rhead[\fancyplain{}{\bfseries
INTRODUZIONE}]{\fancyplain{}{\bfseries\thepage}}
\lhead[\fancyplain{}{\bfseries\thepage}]{\fancyplain{}{\bfseries
INTRODUZIONE}}

Il lavoro da me svolto consiste nell'implementazione dell'algoritmo di type inference di Haskell in ELPI.\\

\section{Stato dell'arte}

\subsection{Premesse}
Iniziamo questa trattazione ponendo alcune basi. Esse faciliteranno la comprensione degli argomenti successivi.\\

\begin{description}
 \item[Haskell] Haskell è un linguaggio di programmazione che adotta il paradigma di programmazione funzionale. Al suo interno è presente il lambda calcolo e il meccanismo delle type class; è presente inoltre una parte più ampia che corrisponde alle librerie.\\
 \item[Type inference] La type inference è il rilevamento automatico del tipo di dato di un'espressione in un linguaggio di programmazione. La capacità di dedurre i tipi automaticamente semplifica molte attività di programmazione, lasciando il programmatore libero di omettere le annotazioni sui tipi pur consentendo il type check.\\
 \item[ELPI] ELPI è un linguaggio di programmazione logico. Esso è un'estensione con vincoli del linguaggio $\lambda$Prolog, il quale a sua volta è un'estensione di Prolog a una logica di ordine superiore.\\
 \item[Prolog] Prolog è un linguaggio di programmazione che adotta il paradigma di programmazione logica. Si basa sul calcolo dei predicati (logica del prim'ordine); la sintassi è composta da formule dette clausole che sono disgiunzioni di letterali del prim'ordine. L'esecuzione di un programma Prolog è comparabile alla dimostrazione di un teorema mediante la regola di inferenza detta risoluzione (essa permette di passare da un numero finito di proposizioni assunte come premesse a una proposizione che funge da conclusione). I concetti fondamentali di questo linguaggio sono l'unificazione, la ricorsione in coda e il backtracking.\\
 \item[$\lambda$Prolog] $\lambda$Prolog è, come già detto, un'estensione di Prolog. Le caratteristiche principali in aggiunta, rispetto a Prolog, sono il polimorfismo, la programmazione di ordine superiore e il lambda calcolo tipato.\\
\end{description}

\subsection{Limiti di $\lambda$Prolog}
Non essendo $\lambda$Prolog un linguaggio di programmazione con vincoli risulta impossibile implementare la type inference mediante esso. I limiti dell'utilizzo di tale linguaggio si riscontrano in particolare nel tentativo di codificare il tipaggio per i costrutti del let-in e della type class.\\
L'unica strategia percorribile sarebbe quella di codificare interamente il sistema punto per punto. In tal caso però si creerebbe un sistema estremamente rigido, il che rende tale strategia impraticabile. Ad esempio, infatti, si potrebbe codificare tutto nelle stringhe e svolgere ogni operazione attraverso di esse; a quel punto ogni cosa sarebbe implementabile, il sistema in questione sarebbe infatti Turing completo. Il problema però è il fatto che in tal caso si dovrebbero abbandonare tutte le features del linguaggio $\lambda$Prolog; infatti la rigidezza del sistema creato non consentirebbe l'utilizzo delle stesse, poiché si risulterebbe vincolati alla nuova struttura.\\
Si è reso dunque necessario l'utilizzo di ELPI, la cui maggiore espressività (non nel senso della Turing completezza) permette di svolgere operazioni impossibili da codificare in $\lambda$Prolog. Prendiamo come esempio i due casi indicati precedentemente:
\begin{itemize}
 \item Per codificare il tipaggio del let-in è necessario l'utilizzo del meccanismo mode di ELPI. Infatti esso offre un maggior controllo sugli elementi del codice poiché permette di accorgersi se essi sono delle variabili non istanziate, così da poterle gestire in modo appropriato.
 \item Per codificare il tipaggio della type class è necessario l'utilizzo dei vincoli. Infatti questi possono sussistere anche non totalmente istanziati e quindi permettono, ad esempio, di fissare l'obbligo di appartenenza di una variabile di tipo non istanziata ad un'istanza di type class.
\end{itemize}
Entrambi i requisiti sono caratteristiche presenti in ELPI ma non in $\lambda$Prolog. Risulta dunque evidente la necessità di utilizzare ELPI come linguaggio di programmazione per poter raggiungere lo scopo prefissato.

\section{ELPI}
\subsection{Introduzione al linguaggio}
%ELPI, così come $\lambda$Prolog, è un linguaggio logico di ordine superiore (HOLP language - Higher Order Logic Programming language).
% ~ La differenza sta nell'aggiunta dei vincoli...
% ~ cos'è l'ordine superiore...
% ~ cosa sono i vincoli... o verrà detto nella sottosezione seguente?































\subsection{Sistema di propagazione dei vincoli}
% ...

\section{Haskell}
% ...
\subsection{BNF utilizzata}
% ...
\subsection{Type class}
% ...
\subsection{Let-in}
% ...

\section{Finalità}
% ~ scopo del mio lavoro
% ...

\clearpage{\pagestyle{empty}\cleardoublepage}	% non numera l'ultima pagina sinistra


\chapter{Implementazione}		% crea il capitolo

\lhead[\fancyplain{}{\bfseries\thepage}]{\fancyplain{}{\bfseries\rightmark}}	% imposta l'intestazione di pagina

% ...

\section{STLC}
% ...

\section{Funzioni ricorsive, match}
% ...

\section{Type class, instanza, schema}
% ...

\section{Let-in}
% ...

\clearpage{\pagestyle{empty}\cleardoublepage}		% non numera l'ultima pagina sinistra


\chapter{Conclusioni}		% per fare le conclusioni

% imposta l'intestazione di pagina
\rhead[\fancyplain{}{\bfseries
CONCLUSIONI}]{\fancyplain{}{\bfseries\thepage}}
\lhead[\fancyplain{}{\bfseries\thepage}]{\fancyplain{}{\bfseries
CONCLUSIONI}}

% ...

\section{Riassumendo}
% ...

\section{La mia esperienza con ELPI}
% ...

\section{Il mio lavoro}
% ...

\section{Sviluppi futuri}
% ...
\subsection{Parser}
% ...
\subsection{Testing}
% ...
\subsection{Estensioni}
% ...


%\begin{thebibliography}{90}		% crea l'ambiente bibliografia

%\rhead[\fancyplain{}{\bfseries \leftmark}]{\fancyplain{}{\bfseries
%\thepage}}

%\addcontentsline{toc}{chapter}{Bibliografia}	% aggiunge la voce Bibliografia nell'indice
%%\addcontentsline{toc}{chapter}{\numberline{}{Bibliografia}}	% provare anche questo comando

%\bibitem{K1} Primo oggetto bibliografia.
%\bibitem{K2} Secondo oggetto bibliografia.
% ...

%\end{thebibliography}

\clearpage{\pagestyle{empty}\cleardoublepage}	% non numera l'ultima pagina sinistra


\chapter*{Ringraziamenti}

\thispagestyle{empty}

% ...


\end{document}
