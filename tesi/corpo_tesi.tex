% 12pt: grandezza carattere
% a4paper: formato a4
% openright: apre i capitoli a destra
% twoside: serve per fare un documento fronteretro
% report: stile tesi (oppure book)
\documentclass[12pt,a4paper,openright,twoside]{report}

\usepackage[italian]{babel} % libreria per scrivere in italiano
\usepackage[latin1]{inputenc}   % libreria per accettare i caratteri digitati da tastiera come 'è' o 'à'
\usepackage{fancyhdr}   % libreria per impostare il documento
\usepackage{indentfirst}    % libreria per avere l'indentazione all'inizio dei capitoli
%\usepackage{showkeys}  % libreria per mostrare le etichette
\usepackage{graphicx}   % libreria per inserire grafici
\usepackage{newlfont}   % libreria per utilizzare font particolari ad esempio \textsc{}
% librerie matematiche
\usepackage{amssymb}
\usepackage{amsmath}
\usepackage{latexsym}
\usepackage{amsthm}

















\oddsidemargin=30pt \evensidemargin=20pt%impostano i margini
%\hyphenation{sil-la-ba-zio-ne pa-ren-te-si}%serve per la sillabazione: tra parentesi 
                       %vanno inserite come nell'esempio le parole 
%                      %che latex non riesce a tagliare nel modo giusto andando a capo.

%
%%%%%%%%%%%%%%%%%%%%%%%%%%%%%%%%%%%%%%%%%comandi per l'impostazione
                                        %   della pagina, vedi il manuale
                                        %   della libreria fancyhdr
                                        %   per ulteriori delucidazioni
\pagestyle{fancy}\addtolength{\headwidth}{20pt}
\renewcommand{\chaptermark}[1]{\markboth{\thechapter.\ #1}{}}
\renewcommand{\sectionmark}[1]{\markright{\thesection \ #1}{}}
\rhead[\fancyplain{}{\bfseries\leftmark}]{\fancyplain{}{\bfseries\thepage}}
\cfoot{}
%%%%%%%%%%%%%%%%%%%%%%%%%%%%%%%%%%%%%%%%%
\linespread{1.3}                        %comando per impostare l'interlinea
%%%%%%%%%%%%%%%%%%%%%%%%%%%%%%%%%%%%%%%%%definisce nuovi comandi
%
\begin{document}
\pagenumbering{roman}                   %serve per mettere i numeri romani
\chapter*{Introduzione}                 %crea l'introduzione (un capitolo
                                        %   non numerato)
%%%%%%%%%%%%%%%%%%%%%%%%%%%%%%%%%%%%%%%%%imposta l'intestazione di pagina
\rhead[\fancyplain{}{\bfseries
INTRODUZIONE}]{\fancyplain{}{\bfseries\thepage}}
\lhead[\fancyplain{}{\bfseries\thepage}]{\fancyplain{}{\bfseries
INTRODUZIONE}}
%%%%%%%%%%%%%%%%%%%%%%%%%%%%%%%%%%%%%%%%%aggiunge la voce Introduzione
                                        %   nell'indice
\addcontentsline{toc}{chapter}{Introduzione}
Questa \`e l'introduzione.
%%%%%%%%%%%%%%%%%%%%%%%%%%%%%%%%%%%%%%%%%non numera l'ultima pagina sinistra
\clearpage{\pagestyle{empty}\cleardoublepage}
\tableofcontents                        %crea l'indice
%%%%%%%%%%%%%%%%%%%%%%%%%%%%%%%%%%%%%%%%%imposta l'intestazione di pagina
\rhead[\fancyplain{}{\bfseries\leftmark}]{\fancyplain{}{\bfseries\thepage}}
\lhead[\fancyplain{}{\bfseries\thepage}]{\fancyplain{}{\bfseries
INDICE}}
%%%%%%%%%%%%%%%%%%%%%%%%%%%%%%%%%%%%%%%%%non numera l'ultima pagina sinistra
\clearpage{\pagestyle{empty}\cleardoublepage}
%\listoffigures                          %crea l'elenco delle figure
%%%%%%%%%%%%%%%%%%%%%%%%%%%%%%%%%%%%%%%%%non numera l'ultima pagina sinistra
%\clearpage{\pagestyle{empty}\cleardoublepage}
%\listoftables                           %crea l'elenco delle tabelle
%%%%%%%%%%%%%%%%%%%%%%%%%%%%%%%%%%%%%%%%%non numera l'ultima pagina sinistra
%\clearpage{\pagestyle{empty}\cleardoublepage}
\chapter{Primo Capitolo}                %crea il capitolo
%%%%%%%%%%%%%%%%%%%%%%%%%%%%%%%%%%%%%%%%%imposta l'intestazione di pagina
\lhead[\fancyplain{}{\bfseries\thepage}]{\fancyplain{}{\bfseries\rightmark}}
\pagenumbering{arabic}                  %mette i numeri arabi
Questo \`e il primo capitolo.
%%%%%%%%%%%%%%%%%%%%%%%%%%%%%%%%%%%%%%%%%non numera l'ultima pagina sinistra
\clearpage{\pagestyle{empty}\cleardoublepage}
%%%%%%%%%%%%%%%%%%%%%%%%%%%%%%%%%%%%%%%%%per fare le conclusioni
\chapter*{Conclusioni}
%%%%%%%%%%%%%%%%%%%%%%%%%%%%%%%%%%%%%%%%%imposta l'intestazione di pagina
\rhead[\fancyplain{}{\bfseries
CONCLUSIONI}]{\fancyplain{}{\bfseries\thepage}}
\lhead[\fancyplain{}{\bfseries\thepage}]{\fancyplain{}{\bfseries
CONCLUSIONI}}
%%%%%%%%%%%%%%%%%%%%%%%%%%%%%%%%%%%%%%%%%aggiunge la voce Conclusioni
                                        %   nell'indice
\addcontentsline{toc}{chapter}{Conclusioni} Queste sono le
conclusioni.\\
In queste conclusioni voglio fare un riferimento alla
bibliografia: questo \`e il mio riferimento \cite{K3,K4}.
\begin{thebibliography}{90}             %crea l'ambiente bibliografia
\rhead[\fancyplain{}{\bfseries \leftmark}]{\fancyplain{}{\bfseries
\thepage}}
%%%%%%%%%%%%%%%%%%%%%%%%%%%%%%%%%%%%%%%%%aggiunge la voce Bibliografia
                                        %   nell'indice
\addcontentsline{toc}{chapter}{Bibliografia}
%%%%%%%%%%%%%%%%%%%%%%%%%%%%%%%%%%%%%%%%%provare anche questo comando:
%%%%%%%%%%%\addcontentsline{toc}{chapter}{\numberline{}{Bibliografia}}
\bibitem{K1} Primo oggetto bibliografia.
\bibitem{K2} Secondo oggetto bibliografia.
\bibitem{K3} Terzo oggetto bibliografia.
\bibitem{K4} Quarto oggetto bibliografia.
\end{thebibliography}
%%%%%%%%%%%%%%%%%%%%%%%%%%%%%%%%%%%%%%%%%non numera l'ultima pagina sinistra
\clearpage{\pagestyle{empty}\cleardoublepage}
\chapter*{Ringraziamenti}
\thispagestyle{empty}
Qui possiamo ringraziare il mondo intero!!!!!!!!!!\\
Ovviamente solo se uno vuole, non \`e obbligatorio.
\end{document}
