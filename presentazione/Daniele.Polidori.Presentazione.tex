\documentclass{beamer}

\usetheme{Warsaw}

\title{Implementazione in un linguaggio logico con vincoli di ordine superiore della type inference di Haskell}
\author{Daniele Polidori}
\institute{Alma Mater Studiorum - Università di Bologna\\Corso di Laurea in Informatica}
\date{III Sessione\\Anno Accademico 2018/2019}

% Imposta il template del footline (così da farci entrare il titolo intero)
\setbeamertemplate{footline}{%
  \leavevmode%
  \hbox{\begin{beamercolorbox}[wd=.5\paperwidth,ht=4.5ex,dp=2.125ex,leftskip=.3cm plus1fill,rightskip=.3cm]{author in head/foot}%
    \usebeamerfont{author in head/foot}
    \insertshortauthor
  \end{beamercolorbox}%
  \begin{beamercolorbox}[wd=.5\paperwidth,ht=4.5ex,dp=2.125ex,leftskip=.3cm,rightskip=.3cm plus1fil]{title in head/foot}%
    \usebeamerfont{title in head/foot}
    \parbox{.45\paperwidth}{\inserttitle}
  \end{beamercolorbox}}%
  \vskip0pt%
}

\setbeamertemplate{navigation symbols}{\tiny\insertframenumber/\inserttotalframenumber}     % Inserisce il numero di pagina al posto dei simboli di navigazione
\setbeamercolor{navigation symbols}{fg=black}   % Colora di nero i simboli di navigazione

% Cambia la dimensione del font di 'verbatim'
\makeatletter
\def\verbatim{\scriptsize\@verbatim \frenchspacing\@vobeyspaces \@xverbatim}
\makeatother


\begin{document}


{
\setbeamertemplate{footline}{}      % Elimina il footline in questa slide
\setbeamertemplate{navigation symbols}{}    % Lascia uno spazio vuoto al posto dei simboli di navigazione
\begin{frame}
 \titlepage     % Beamer's \maketitle
\end{frame}
}
%\addtocounter{framenumber}{-1}     % Non considera questa slide nel conteggio delle pagine


\begin{frame}

 \frametitle{Lavoro svolto}
 
 \textbf{Problema da risolvere}: Implementare in un linguaggio logico con vincoli di ordine superiore la type inference di Haskell.
 \begin{itemize}
  \item Codifica della sintassi di Haskell.
  \item Implementazione dell’algoritmo di type inference con le type class di Haskell.
 \end{itemize}
 Linguaggio di programmazione utilizzato: ELPI.

 \vfill
 \textbf{Finalità:}
 \begin{enumerate}
  \item Dimostrare che ELPI è più espressivo di $\lambda$Prolog.
  \item Strumento di prova per testare, implementare e studiare nuove estensioni al meccanismo delle type class di Haskell.\\(Sviluppi futuri)
 \end{enumerate}

\end{frame}


\begin{frame}[fragile=singleslide]      % 'fragile' necessario per usare 'verbatim'
 
 \frametitle{Type class}

 \underline{Esempio}:
 \begin{verbatim}
1.  typeclass printable T [fun_decl print (base (arr T string))].
2.  istanza printable int [fun_impl print (fun print_int)].

goal>  pi x \ of x int => of (app (fun print) x) T1.
%   T1 = string

goal>  pi x \ of x bool => of (app (fun print) x) T2.
%   Failure

goal>  pi x \ of x T3 => of (app (fun print) x) T4.
%   T3 = X0
%   T4 = string
%   istanza printable X0 [...]
 \end{verbatim}

\end{frame}


\begin{frame}[fragile=singleslide]      % 'fragile' necessario per usare 'verbatim'

 \frametitle{ELPI}

 \begin{itemize}
  \item Higher Order constraint Logic Programming language.
  \item Estensione con vincoli del linguaggio $\lambda$Prolog.
 \end{itemize}

 \vfill
 
 \underline{Esempio (STLC)}:
 \begin{verbatim}
 1.  kind tipo type.
 2.  type arr tipo -> tipo -> tipo.
 3.
 4.  kind term type.
 5.  type app term -> term -> term.
 6.  type lam (term -> term) -> term.
 7.
 8.  type of term -> tipo -> prop.
 9.  mode (of i o).
10.  of (uvar _ as X) T :- !, declare_constraint (of X T) [X].
11.  of (app X Y) B :- of X (arr A B), of Y A.
12.  of (lam F) (arr A B) :- pi x \ of x A => of (F x) B.
 \end{verbatim}

\end{frame}


\begin{frame}

 \frametitle{Istanziazione degli schemi}

 % ...

\end{frame}


\begin{frame}

 \frametitle{Let-in: Regola di generalizzazione}

 % ...

\end{frame}


\begin{frame}

 \frametitle{Conclusioni e sviluppi futuri}

 \textbf{Stima del lavoro svolto}:
 \begin{itemize}
  \item Tempo impiegato: 180 ore.
  \item Linee di codice prodotte: 600, scritte in linguaggio ELPI.
 \end{itemize}

 \vfill

 \textbf{Sviluppi futuri}:
 \begin{enumerate}
  \item Parser
  \item Testing
  \item Estensioni
 \end{enumerate}

\end{frame}


\begin{frame}

 \frametitle{Vi ringrazio per l'attenzione}

 Se ci sono domande\ldots

 \vfill

 \begin{flushright}
  \textbf{Presentata da}: Daniele Polidori\\
  \textbf{Relatore}: Professor Claudio Sacerdoti Coen
 \end{flushright}

\end{frame}


\begin{frame}

 \frametitle{}

 % ...

\end{frame}


\begin{frame}

 \frametitle{}

 % ...

\end{frame}


\end{document}
