\documentclass{beamer}

\usetheme{Warsaw}

\title{Implementazione in un linguaggio logico con vincoli di ordine superiore della type inference di Haskell}
\author{Daniele Polidori}
\institute{Alma Mater Studiorum - Università di Bologna\\Corso di Laurea in Informatica}
\date{III Sessione\\Anno Accademico 2018/2019}

% Imposta il template del footline (così da farci entrare il titolo intero)
\setbeamertemplate{footline}{%
  \leavevmode%
  \hbox{\begin{beamercolorbox}[wd=.5\paperwidth,ht=4.5ex,dp=2.125ex,leftskip=.3cm plus1fill,rightskip=.3cm]{author in head/foot}%
    \usebeamerfont{author in head/foot}
    \insertshortauthor
  \end{beamercolorbox}%
  \begin{beamercolorbox}[wd=.5\paperwidth,ht=4.5ex,dp=2.125ex,leftskip=.3cm,rightskip=.3cm plus1fil]{title in head/foot}%
    \usebeamerfont{title in head/foot}
    \parbox{.45\paperwidth}{\inserttitle}
  \end{beamercolorbox}}%
  \vskip0pt%
}

\setbeamertemplate{navigation symbols}{\tiny\insertframenumber/\inserttotalframenumber}     % Inserisce il numero di pagina al posto dei simboli di navigazione
\setbeamercolor{navigation symbols}{fg=black}   % Colora di nero i simboli di navigazione


\begin{document}


{
\setbeamertemplate{footline}{}      % Elimina il footline in questa slide
\setbeamertemplate{navigation symbols}{}    % Lascia uno spazio vuoto al posto dei simboli di navigazione
\begin{frame}
 \titlepage     % Beamer's \maketitle
\end{frame}
}
%\addtocounter{framenumber}{-1}     % Non considera questa slide nel conteggio delle pagine


\begin{frame}

 \frametitle{Lavoro svolto}
 
 \textbf{Problema da risolvere}: Implementare in un linguaggio logico con vincoli di ordine superiore la type inference di Haskell.
 \begin{itemize}
  \item Codifica della sintassi di Haskell.
  \item Implementazione dell’algoritmo di type inference con le type class di Haskell.
 \end{itemize}
 Linguaggio di programmazione utilizzato: ELPI.

 \vfill
 \textbf{Finalità:}
 \begin{enumerate}
  \item Dimostrare che ELPI è più espressivo di $\lambda$Prolog.
  \item Strumento di prova per testare, implementare e studiare nuove estensioni al meccanismo delle type class di Haskell.\\(Sviluppi futuri)
 \end{enumerate}

\end{frame}


\begin{frame}
 
 \frametitle{Type class}
 
 % ...
 
\end{frame}


\begin{frame}

 \frametitle{ELPI}

 \begin{itemize}
  \item Higher Order constraint Logic Programming language.
  \item Estensione con vincoli del linguaggio $\lambda$Prolog.
 \end{itemize}

\end{frame}


\begin{frame}

 \frametitle{Ricerca di type class}

 % ...

\end{frame}


\begin{frame}

 \frametitle{Let-in: Regola di generalizzazione}

 % ...

\end{frame}


\begin{frame}

 \frametitle{Conclusioni e sviluppi futuri}

 Stima del lavoro svolto:
 \begin{itemize}
  \item Tempo impiegato: 180 ore.
  \item Linee di codice prodotte: 600, scritte in linguaggio ELPI.
 \end{itemize}

 \vfill

 Sviluppi futuri:
 \begin{enumerate}
  \item Parser
  \item Testing
  \item Estensioni
 \end{enumerate}

\end{frame}


\begin{frame}

 \frametitle{Vi ringrazio per l'attenzione}

 Se ci sono domande\ldots
 
 \vfill
 \begin{flushright}
  Presentata da: Daniele Polidori\\
  Relatore: Professor Claudio Sacerdoti Coen
 \end{flushright}

\end{frame}


\begin{frame}

 \frametitle{}

 % ...

\end{frame}


\begin{frame}

 \frametitle{}

 % ...

\end{frame}


\end{document}
